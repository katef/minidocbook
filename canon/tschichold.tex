
% Jan Tschichold
% intended for 2:3

% XXX: no idea where to put a footer

\startenvironment tschichold

	\startMPdefinitions
		vardef canon(expr h, w) =
			path p, q;
			path m;
			pair r[];

			p := (w, 0) -- (0, h); % diagonal
			hyp := h ++ w; % length of p
			if w < 0:
				hyp := -hyp;
			fi

			z0 = (-w + hyp, 0); % hyp rotated down to bottom edge

			% Amazing that this is exactly the arc I want. It's equivalent to
			% rotating (-w, 0) -- (-w + hyp, 0) until it touches (0, h)
			% which is the same arc from a segment of the circle:
			%   path c; c := fullcircle scaled 2hyp shifted (-w, 0);
			%   draw subpath (0, 2) of c cutafter (0, h) -- (w, h);
			m := z0{up} .. (0, h);

			z1 = m intersectionpoint p;

			% Tschichold specifies margin ratio 1:1:2:3
			% TODO: use array of margin[] perhaps
			rhs := w - xpart z1;
			lhs := rhs / 2;
			ths := abs(lhs); % abs for when w < 0
			bhs := ypart z1;

			draw m withcolor blue;
			draw p withcolor blue;

			r0 = (0 + lhs, h - ths); % top left
			r1 = (w - rhs, h - ths); % top right
			r2 = (w - rhs, 0 + bhs); % bottom right
			r3 = (0 + lhs, 0 + bhs); % bottom left

			r
		enddef;
	\stopMPdefinitions

\stopenvironment


