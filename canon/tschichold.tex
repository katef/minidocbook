
% Jan Tschichold
% intended for 2:3

% XXX: no idea where to put a footer

\startenvironment tschichold

	% TODO: set double page here; it's a default for this canon
	% but a theme could override that for a single page layout,
	% if it were really appropriate
	\setuppagenumbering[alternative=doublesided]

	\startMPdefinitions
		def canon(expr h, w) =
			path p, q;
			path m;

			p := (w, 0) -- (0, h); % diagonal
			hyp := h ++ w; % length of p
			if w < 0:
				hyp := -hyp;
			fi

			z0 = (-w + hyp, 0); % hyp rotated down to bottom edge

			% Amazing that this is exactly the arc I want. It's equivalent to
			% rotating (-w, 0) -- (-w + hyp, 0) until it touches (0, h)
			% which is the same arc from a segment of the circle:
			%   path c; c := fullcircle scaled 2hyp shifted (-w, 0);
			%   draw subpath (0, 2) of c cutafter (0, h) -- (w, h);
			m := z0{up} .. (0, h);

			z1 = m intersectionpoint p;

			% Tschichold specifies margin ratio 1:1:2:3
			% TODO: use array of margin[] perhaps
			rhs := w - xpart z1;
			lhs := rhs / 2;
			ths := abs(lhs); % abs for when w < 0
			bhs := ypart z1;

			draw m withcolor blue;
			draw p withcolor blue;

			draw (w - rhs, 0) -- (w - rhs, h) withcolor red;
			draw (0, h - ths) -- (w, h - ths) withcolor red;
			draw (0, 0 + bhs) -- (w, 0 + bhs) withcolor red;
			draw (0 + lhs, 0) -- (0 + lhs, h) withcolor red;
		enddef;
	\stopMPdefinitions

	\startreusableMPgraphic{canon-recto}
		h := \overlayheight;
		w := \overlaywidth;
		canon(h, w);
	\stopreusableMPgraphic

	\startreusableMPgraphic{canon-verso}
		h := \overlayheight;
		w := \overlaywidth;
		canon(h, -w);
	\stopreusableMPgraphic

	\startmode[debug]
		\defineoverlay[canon-recto] [\reusableMPgraphic{canon-recto}]
		\defineoverlay[canon-verso] [\reusableMPgraphic{canon-verso}]

		\setupbackgrounds[leftpage] [background=canon-verso]
		\setupbackgrounds[rightpage][background=canon-recto]
	\stopmode

	\startMPcalculation
		h := PaperHeight;
		w := PaperWidth;
		canon(h, w);

		passvariable("canon:textleft",   lhs);
		passvariable("canon:textright",  w - rhs);
		passvariable("canon:textwidth",  w - rhs - lhs);
		passvariable("canon:texttop",    ths);
		passvariable("canon:textbottom", bhs);
		passvariable("canon:textheight", h - ths - bhs);
	\stopMPcalculation

	\setuplayout[
		header=\bodyfontsize,
		headerdistance=\bodyfontsize, % XXX: this ought to be found geometrically
		top=0pt,
		topspace=\dimexpr\MPrunvar{canon:texttop}bp-\headerheight-\headerdistance\relax,
		footer=0pt,
		footerdistance=0pt,
		height=\dimexpr\MPrunvar{canon:textheight}bp
			+\headerheight+\headerdistance
			+\footerheight+\footerdistance\relax,
		backspace=\MPrunvar{canon:textleft}bp,
		margin=0pt, % TODO: find where to put this
		width=\MPrunvar{canon:textwidth}bp,
	]

\stopenvironment


