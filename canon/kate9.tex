
% my other attempt at a rational single-page geometry
% intended to be suitable for any ratio

\startenvironment kate9

	\startMPdefinitions
		vardef canon(expr h, w) =
			path p, q, g, d;
			path m, n, o;
			path t, v, u;
			path c;
			path rhs, bhs;
			pair r[];

			p := (w, 0) -- (0, .5h); % bottom half-page diagonal
			g := (w, h) -- (0, 0); % one-page diagonal

			draw (0, 0) -- (w, h) withcolor blue;

			draw subpath (0, 1) of fullcircle scaled 2h shifted (w - h, 0) withcolor blue;

			draw (w, 0) -- (0, h) withcolor blue;

			c = (w, 0) -- (0, h);

			z0 = p intersectionpoint g; % centre point

			m := (xpart z0, 0) -- (xpart z0, h); % vertical
			n := (w, 0) -- (xpart z0, h); % left diagonal
			o := z0 -- (-xpart z0, 0); % projection off edge

			z1 = o intersectionpoint ((0, 0) -- (0, h)); % edge

			t := z1 -- (xpart z0, 0); % crossbar

			q := (0, h) -- (w, ypart z0); % half-page diagonal
			d := (w, h) -- (0, ypart z0); % half-page diagonal

			z5 = (subpath (0, 1) of fullcircle scaled 2h shifted (w - h, 0)) intersectionpoint ((0, h) -- (w, ypart z0));
			rhs := (xpart z5, 0) -- (xpart z5, h);

			draw (0, ypart z0) -- (w, ypart z0) withcolor blue;

			z3 = rhs intersectionpoint c;

			z4 = rhs intersectionpoint d; % top right

			bhs := (0, ypart z3) -- (w, ypart z3);

			z2 = bhs intersectionpoint g; % bottom left

			draw p withcolor blue;
			draw q withcolor blue;
			draw d withcolor blue;

			r0 = (xpart z2, ypart z4); % top left
			r1 = (xpart z3, ypart z3); % bottom right

			r
		enddef;

	\stopMPdefinitions

\stopenvironment

