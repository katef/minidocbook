
% makes the most sense for ISO 216 A series, where the text area is the next number in the series
% does not work on 1:1 (occupies entire page)
% I don't know if there's a proper name for this canon.
% http://www.oert.org/en/typography-and-proportions/

\startenvironment fib

	\setuppagenumbering[alternative=doublesided]

	\startMPdefinitions
		vardef canon(expr h, w) =
			path p, q;
			path r[];

			p := (w,  0) -- (0, h); % one-page diagonal
			q := (-w, 0) -- (w, h); % two-page diagonal
			z0 = q intersectionpoint ((0,0) -- (0, h)); % point on edge

			z1 = whatever[( w, 0), (0, h)]; % bottom right
			z2 = whatever[(-w, 0), (w, h)]; % top right

			xpart z1 = xpart z2;
			ypart z2 - ypart z1 = abs(w);

			z3 = p intersectionpoint ((w, ypart z2) -- (0, ypart z2)); % top left

			z4 = (xpart z3, ypart z1); % bottom left

			draw p withcolor blue;
			draw z0 -- (w, h) withcolor blue;

			draw fullcircle scaled abs(w) shifted (.5w, (ypart z2 - ypart z1) / 2 + ypart z1) withcolor blue;

%			draw (xpart z1, 0) -- (xpart z1, h) withcolor red;
%			draw (xpart z3, 0) -- (xpart z3, h) withcolor red;
%			draw (0, ypart z1) -- (w, ypart z1) withcolor red;
%			draw (0, ypart z2) -- (w, ypart z2) withcolor red;

%			draw (0, .5h) -- (w, 0) withcolor green; % maybe for a footer

			r0 = (xpart z3, ypart z3); % top left
			r1 = (xpart z2, ypart z2); % top right
			r2 = (xpart z1, ypart z1); % bottom right
			r3 = (xpart z3, ypart z1); % bottom left

			r
		enddef;
	\stopMPdefinitions

\stopenvironment

