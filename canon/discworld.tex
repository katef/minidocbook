
% layout of Corgi editions of Terry Pratchett’s Discworld books
% https://eegg.wordpress.com/2010/01/25/page-margins-in-principle-and-practice/
% intended for 2:3
% no header, only footer

\startenvironment discworld

	\setuppagenumbering[alternative=doublesided]

	\startMPdefinitions
		def canon(expr h, w) =
			path p, q, r;
			path m, n, o;
			path t, v, u;

			p := (w,  0) -- (0, .5h); % bottom half-page diagonal
			q := (w,  h) -- (0, .5h); % top half-page diagonal
			r := (w,  h) -- (0, 0); % one-page diagonal

			z0 = p intersectionpoint r; % centre point

			m := (xpart z0, 0) -- (xpart z0, h); % vertical
			n := (w, 0) -- (xpart z0, h); % left diagonal
			o := z0 -- (-xpart z0, 0); % projection off edge

			z1 = o intersectionpoint ((0, 0) -- (0, h)); % edge

			t := z1 -- (xpart z0, 0); % crossbar

			z2 = t intersectionpoint r; % bottom left

			v := (0, ypart z2) -- (w, ypart z2); % bottom

			z3 = v intersectionpoint n; % bottom right

			u := (xpart z3, 0) -- (xpart z3, h); % right

			z4 = u intersectionpoint r; % top right

			draw p withcolor blue;
			draw q withcolor blue;
			draw r withcolor blue;
			draw m withcolor blue;
			draw n withcolor blue;
			draw z0 -- z1 withcolor blue;
			draw t withcolor blue;

			draw (xpart z2, 0) -- (xpart z2, h) withcolor red;
			draw (xpart z3, 0) -- (xpart z3, h) withcolor red;
			draw (0, ypart z2) -- (w, ypart z2) withcolor red;
			draw (0, ypart z4) -- (w, ypart z4) withcolor red;

			% TODO: just a guess for the footer
%			draw (0, ypart z1 / 2) -- (xpart z0, 0) withcolor green;

		enddef;
	\stopMPdefinitions

	\startreusableMPgraphic{canon-recto}
		h := \overlayheight;
		w := \overlaywidth;
		canon(h, w);
	\stopreusableMPgraphic

	\startreusableMPgraphic{canon-verso}
		h := \overlayheight;
		w := \overlaywidth;
		canon(h, -w);
	\stopreusableMPgraphic

	\startmode[debug]
		\defineoverlay[canon-recto] [\reusableMPgraphic{canon-recto}]
		\defineoverlay[canon-verso] [\reusableMPgraphic{canon-verso}]

		\setupbackgrounds[leftpage] [background={canon-verso, footer-verso}]
		\setupbackgrounds[rightpage][background={canon-recto, footer-recto}]
	\stopmode

	\startMPcalculation
		h := PaperHeight;
		w := PaperWidth;
		canon(h, w);

		passvariable("canon:textleft",   xpart z2);
		passvariable("canon:textright",  w - xpart z3);
		passvariable("canon:textwidth",  xpart z3 - xpart z2);
		passvariable("canon:texttop",    h - ypart z4);
		passvariable("canon:textbottom", ypart z2);
		passvariable("canon:textheight", ypart z4 - ypart z2);
	\stopMPcalculation

	\setuplayout[
		header=0pt,
		headerdistance=0pt,
		top=0pt,
		topspace=\dimexpr\MPrunvar{canon:texttop}bp-\headerheight-\headerdistance\relax,
		footer=\bodyfontsize,
		footerdistance=\bodyfontsize, % XXX: this ought to be found geometrically
		height=\dimexpr\MPrunvar{canon:textheight}bp
			+\headerheight+\headerdistance
			+\footerheight+\footerdistance\relax,
		backspace=\MPrunvar{canon:textleft}bp,
		margin=0pt, % TODO: find where to put this
		width=\MPrunvar{canon:textwidth}bp,
	]

\stopenvironment

