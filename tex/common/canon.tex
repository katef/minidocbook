
\startMPcalculation
	h := PaperHeight;
	w := PaperWidth;

	r := canon(h, w); % top left, top right, bottom right, bottom left

	passvariable("canon:textleft",   xpart r0);
	passvariable("canon:textright",  w - xpart r1);
	passvariable("canon:textwidth",  xpart r1 - xpart r0);
	passvariable("canon:texttop",    h - ypart r0);
	passvariable("canon:textbottom", ypart r2);
	passvariable("canon:textheight", ypart r0 - ypart r2);
\stopMPcalculation

% The default is single-sided where the left and right margins are equal.
% This may be overridden by explicitly specifying singlesided or doublesided instead.
%
% 1pt is our threshold for considering margins equal. This caters for inaccuracies
% where the left and right margins are calculated by different means.
\startnotallmodes[singlesided,doublesided]
	\ifabsdim\dimexpr\MPrunvar{canon:textleft}bp-\MPrunvar{canon:textright}bp\relax<1pt
		\enablemode[singlesided]
	\else
		\enablemode[doublesided]
	\fi
\stopnotallmodes

\startmode[debug]

	\startmode[singlesided,doublesided]
		\startreusableMPgraphic{canon-recto}
			h := \overlayheight;
			w := \overlaywidth;

			r := canon(h, w);

			draw r0 -- r1 -- r2 -- r3 -- cycle withcolor red;

%			draw r0 withpen pencircle scaled 10bp withcolor red;
%			draw r1 withpen pencircle scaled 10bp withcolor red;
%			draw r2 withpen pencircle scaled 10bp withcolor red;
%			draw r3 withpen pencircle scaled 10bp withcolor red;

%			draw (xpart r1, 0) -- (xpart r1, h) withcolor red; % right
%			draw (xpart r0, 0) -- (xpart r0, h) withcolor red; % left
%			draw (0, ypart r0) -- (w, ypart r0) withcolor red; % top
%			draw (0, ypart r2) -- (w, ypart r2) withcolor red; % bottom
		\stopreusableMPgraphic

		\defineoverlay[canon-recto] [\reusableMPgraphic{canon-recto}]
	\stopmode

	\startmode[doublesided]
		\startreusableMPgraphic{canon-verso}
			h := \overlayheight;
			w := \overlaywidth;

			r := canon(h, -w);

			draw r0 -- r1 -- r2 -- r3 -- cycle withcolor red;

%			draw r0 withpen pencircle scaled 10bp withcolor red;
%			draw r1 withpen pencircle scaled 10bp withcolor red;
%			draw r2 withpen pencircle scaled 10bp withcolor red;
%			draw r3 withpen pencircle scaled 10bp withcolor red;

%			draw (xpart r1, 0) -- (xpart r1, h) withcolor red; % right
%			draw (xpart r0, 0) -- (xpart r0, h) withcolor red; % left
%			draw (0, ypart r0) -- (w, ypart r0) withcolor red; % top
%			draw (0, ypart r2) -- (w, ypart r2) withcolor red; % bottom
		\stopreusableMPgraphic

		\defineoverlay[canon-verso] [\reusableMPgraphic{canon-verso}]
	\stopmode

	\startmode[singlesided]
		\setupbackgrounds[page]     [background=canon-recto]
	\stopmode
	\startmode[doublesided]
		\setupbackgrounds[leftpage] [background=canon-verso]
		\setupbackgrounds[rightpage][background=canon-recto]
	\stopmode

\stopmode

\startmode[singlesided]
	\setuppagenumbering[alternative=singlesided]
\stopmode
\startmode[doublesided]
	\setuppagenumbering[alternative=doublesided]
\stopmode

\setuplayout[
	margin=0pt, % TODO: find where to put this
	header=0pt,
	headerdistance=0pt,
	top=0pt,
	topspace=\dimexpr\MPrunvar{canon:texttop}bp
		-\headerheight-\headerdistance\relax,
	height=\dimexpr\MPrunvar{canon:textheight}bp
		+\headerheight+\headerdistance
		+\footerheight+\footerdistance\relax,
	backspace=\MPrunvar{canon:textleft}bp,
	width=\MPrunvar{canon:textwidth}bp
]

% XXX: this ought to be found geometrically
\setuplayout[
	footer=\bodyfontsize,
	footerdistance=\bodyfontsize
]

