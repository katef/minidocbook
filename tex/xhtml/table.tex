
\startxmlsetups xhtml:setup-table
	\xmlsetsetup{\xmldocument}{table|tbody|thead}{xhtml:*}
	\xmlsetsetup{\xmldocument}{td|th}{xhtml:*}
	\xmlsetsetup{\xmldocument}{tbody/tr}{xhtml:*}
	\xmlsetsetup{\xmldocument}{thead/tr}{xhtml:*}
\startnotmode[minidocbook]
	\xmlsetsetup{\xmldocument}{table/tr}{xhtml:tr} % dissallowed in minidocbook's DTD
\stopnotmode
\stopxmlsetups

\xmlregistersetup{xhtml:setup-table}

% XXX: is this used?
\startxmlsetups xhtml:col
	c
\stopxmlsetups

\startxmlsetups xhtml:table
%local function ForLorien(n,maxa,maxb,answers)
%context.startcolumns { n = 3 }
%context.starttabulate { "|r|c|r|c|r|" }
%for i=1,n do
%local sign = random(0,1) > 0.5
%local a, b = random(1,maxa or 99), random(1,max or maxb or 99)
%if b > a and not sign then a, b = b, a end
%context.NC()
%context(a)
%context.NC()
%context.mathematics(sign and "+" or "-")
%context.NC()
%context(b)
%context.NC()
%context("=")
%context.NC()
%context(answers and (sign and a+b or a-b))
%context.NC()
%context.NR()
%end
%context.stoptabulate()
%context.stopcolumns()
%context.page()
%end

% XXX: i can't find a way to use this:
% \xmlconcat{#1}{col}{\|}
% TODO: perhaps i could do this from inside lua

	% note for HTML this includes <tr> directly, whereas for minidocbook
	% we always require <tbody>. Alternatively we could pre-process HTML
	% into shape, but I'd rather avoid the extra step.

	\starttabulate[|l|l|l|l|l|l|l|]
	\xmlall{#1}{thead|tbody|tr}
	\stoptabulate
\stopxmlsetups

\startxmlsetups xhtml:thead
	\xmlflush{#1}
\stopxmlsetups

\startxmlsetups xhtml:tbody
	\xmlflush{#1}
\stopxmlsetups

\startxmlsetups xhtml:tr
	\NC
	\xmlflush{#1}
	\NR
\stopxmlsetups

\startxmlsetups xhtml:th
	% TODO: use <col/> for alignment
	\bf \xmlflush{#1}
	\NC
\stopxmlsetups

\startxmlsetups xhtml:td
	% TODO: rowspan
	\doifnumberelse {\xmlatt{#1}{rowspan}}
		{\xmlflush{#1}}
		{\xmlflush{#1}}
	\NC
\stopxmlsetups

